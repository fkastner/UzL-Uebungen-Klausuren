% !TeX spellcheck = de_DE
\documentclass[12pt]{article}

\usepackage[a4paper,textwidth=16cm,bottom=1.5cm]{geometry}
\usepackage[T1]{fontenc}
\usepackage[utf8]{inputenc}
\usepackage[ngerman]{babel}

\usepackage[klausur]{aufgabenzettel}

%%%%%%%%%%%%%%%
\renewcommand*{\arraystretch}{1.4}
\setlength{\parindent}{0pt}
\pagestyle{empty}
%%%%%%%%%%%%%%%

%\vorlesung{Numerische Mathematik II für MML}
%\datum{\today}

%%%%%%%%%%%%%%%

\mitLoesung

%%%%%%%%%%%%%%%
\begin{document}
\maketitle
\thispagestyle{empty}
%
Bitte \underline{alle Bl\"atter} mit Namen versehen, fortlaufend
nummerieren und am Schluss der Klausur in die Aufgabenbl\"atter
einlegen. Sie werden mit diesen zusammen abgegeben. Beginnen Sie
\underline{jede Aufgabe mit einem neuen Blatt}. Bei der Bearbeitung
der Aufgaben m\"ussen alle verwendeten S\"atze und Verfahren, die
erforderlichen Voraussetzungen, die Rechenwege und s\"amtliche
Zwischenergebnisse angegeben werden.
%
\begin{itemize}
  \item Sie haben 90 Minuten Zeit zum Bearbeiten der Klausur.
  \item Zugelassene Hilfsmittel sind nur das Formelblatt, welches der Klausur beiliegt
  (keine Taschenrechner, elektronische Ger\"ate aller Art, Handys, B\"ucher, \"Ubungsaufgaben, Beispielrechnungen, etc.).
  \item Die Klausur wird nur gewertet, wenn Sie die Voraussetzungen zur Teilnahme an dieser
  Klausur erf\"ullt haben.
  \item Alle Bl\"atter inklusive der Aufgabenbl\"atter und des Formelblattes m\"ussen am
  Ende der Bearbeitungszeit mit der Klausur abgegeben werden.
\end{itemize}
%
\vspace{1cm}
%
\punkteblock[0.8\textwidth]{8,11,11,12}
%
\newpage
%
%
% -----------------------------------------------------
% -----------------AUFGABEN----------------------------
%
%
\begin{aufgabe}[8 Punkte]
    Kreuzen Sie bitte an, ob die jeweilige Aussage wahr (W) oder falsch (F) ist. F\"ur jedes
    korrekt gesetzte Kreuz gibt es 0.5 Punkte, f\"ur jedes nicht korrekt gesetzte Kreuz werden 0.5 Punkte
    abgezogen. Ist die Summe der erzielten Punkte negativ, so wird die gesamte Aufgabe mit
    0 Punkten bewertet.
    \begin{enumerate}[a)]
        \item Frage
        
        \begin{ankreuzblock}
            \antwort{W}{Antwort}
            \antwort{F}{Antwort}
            \antwort{F}{Antwort}
            \antwort{W}{Antwort}
        \end{ankreuzblock}
        \item Frage
        
        \begin{ankreuzblock}
            \antwort{W}{Antwort}
            \antwort{F}{Antwort}
            \antwort{F}{Antwort}
            \antwort{W}{Antwort}
        \end{ankreuzblock}
        \item Frage
        
        \begin{ankreuzblock}
            \antwort{W}{Antwort}
            \antwort{F}{Antwort}
            \antwort{F}{Antwort}
            \antwort{W}{Antwort}
        \end{ankreuzblock}
        \item Frage
        
        \begin{ankreuzblock}
            \antwort{W}{Antwort}
            \antwort{F}{Antwort}
            \antwort{F}{Antwort}
            \antwort{W}{Antwort}
        \end{ankreuzblock}
    \end{enumerate}
\end{aufgabe}
%
%
% -----------------------------------------------------
% -----------------------------------------------------
%
%
\newpage
%
\begin{aufgabe}[11 Punkte]
    content...
\end{aufgabe}
%
\begin{loesung}
    content...
\end{loesung}
%
%
% -----------------------------------------------------
% -----------------------------------------------------
%
%
\begin{aufgabe}[11 Punkte]
    content...
\end{aufgabe}
%
\begin{loesung}
    content...
\end{loesung}
%
%
% -----------------------------------------------------
% -----------------------------------------------------
%
%
\begin{aufgabe}[12 Punkte]
    content...
\end{aufgabe}
%
\begin{loesung}
    content...
\end{loesung}
%
%
% -----------------------------------------------------
% -----------------------------------------------------
%
%
\end{document}
